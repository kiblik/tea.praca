\chapter{Teoretické východiská práce}

Pre správne smerovanie práce je potrebné uviesť východiska, pre ktoré tu bol dôvod vzniku výsledného produktu.
Na jednej strane sú to požiadavky na vzdelávanie z pohľadu štátu (kapitola \ref{svp}), ďalej sa pozrieme psychologické aspekty z pohľadu schopnosti poznávania a chápania (kapitola \ref{piaget}) a tiež na zo strany kognitívno-poznávacieho hľadiska (kapitola \ref{bloom}).

Taktiež sa zhrnieme podobné programy (kapitola \ref{podobne}), aby sme ukázali prínos tejto práce, a nakoľko sa nejedná o nové prostredie, ale jeho inováciu, uvedieme aktuálny stav a dôvody inovovania (kapitola \ref{karelnagjh}).

\section{Štátny vzdelávací program} \label{svp}

Nostným dokumentom určujúcim vzdelávanie v Slovenskej republike (samozrejme okrem zákonu č. 245/2008 Z.z. o výchove a vzdelávaní, jeho dodatkoch a vyhláškach Ministerstva školstva Slovenskej republiky) je Štátny vzdelávací program.
Na stránke štátneho vzdelávacieho ústavu \cite{statpedu,svp} je definovaný nasledovne:
,,Štátny vzdelávací program je záväzný dokument, ktorý stanovuje všeobecné ciele vzdelávania a kľúčové kompetencie, ku ktorým má vzdelávanie smerovať. 
Ciele vzdelávania sú postavené tak, aby sa zabezpečil vyvážený rozvoj osobnosti žiakov. 
Štátny vzdelávací program vymedzuje aj rámcový obsah vzdelávania. 
Je východiskom pre tvorbu školského vzdelávacieho programu, v ktorom sa zohľadňujú aj špecifické podmienky a potreby regiónu. 
Štátny vzdelávací program vydáva a zverejňuje pre jednotlivé stupne vzdelania Ministerstvo školstva, vedy, výskumu a športu Slovenskej republiky.''

ŠVP\footnote{Štátny vzdelávací program} okrem iného určuje minimálny počet hodín, pre jednotlivé predmety, ktoré sa majú odučiť.

ŠVP sa ďalej rozdeľuje podľa stupňov, na ktorých sa uplatňuje počnúc materskými škola končiac strednými školami.
Následne sa okrem iného člení podľa vzdelávacích oblastí a predmetov v nich.
Každý predmet je naplnený vzdelávacím obsahom.
Ten je pre informatiku na každom stupni delený piatich tematických okruhov.
Sú to:
\begin{itemize}
    \item Informácie okolo nás
    \item Komunikácia prostredníctvom IKT
    \item Postupy, riešenie problémov, algoritmické myslenie
    \item Princípy fungovania IKT
    \item Informačná spoločnosť
\end{itemize}

Ich hlbšiu analýzu nájdeme v dokumente \cite{ISCED2}.

\todo[inline]{som pojde este popis dokument a toho, ze preco karel- algoritmy adt, to ze tam nie su cykly, ale ze na to su aj ine jazyky a tak}

ŠVP je hlavným dokumentom popisujúci minimálne požiadavky na obsah hodín respektíve na zoznam toho, čo by mali žiaci vedieť.
Ďalším dokumentom, ktorý sa obsahovo odráža od ŠVP je školský vzdelávací program.
Jeho obsah je volený zväčša predmetovou komisiou na každej škole a preto nemusí byť obsah učiva na všetkých školách rovnaký.
Zatiaľ čo ŠVP špecifikuje čo má žiak ovládať na konci stupňa, školský vzdelávací program špecifikuje, ktorej časti sa koľko budu vyučujúci venovať v jednotlivých ročníkoch.
Zákon umožňuje ísť obsahovo aj nad rámec ŠVP a taktiež umožňuje riaditeľom škôl aká hodinová dotácia bude na predmet v jednotlivých ročníkoch pre určité predmety.

\section{Piagetova teória kognitívneho vývinu} \label{piaget}

Ako sa dozvedáme v \cite{veselsky,piaget1976} Jean Piaget člení kognitívny vývin človeka do štyroch fáz.
\begin{itemize}
    \item Senzomotorické štádium \\
Dieťa sa v ňom nachádza od narodenia do približne dvoch rokov.
V tomto období sa objavuje zámerne konanie, začína chápať existencia predmetov pretrváva aj potom, čo prestatú pôsobiť na zmysly, učia sa pokusom a omylom a ku koncu tohto štádia je ich správanie premyslenejšie a plánovitejšie. 
    \item Predoperačné štádium\\
Vek, v ktorom je dieťa v tom to štádiu, je od konca senzomotorického štádia to približne nástupu na základnú školu (sedem rokov).
V tomto čase si osvojujú jazyk a využívajú mentálne reprezentácia.
Sú schopné s časovým odstupom napodobňovať už videné modely.
Ich myslenie je pred-logické, intuitívne, je závisle na aktuálnom vnímaní.
Ešte u nich zlyháva riešenie úloh založených na princípe stálosti množstva, objemu, hmotnosti, dĺžky či počtu.
Prejavuje sa u nich egocentrizmus v ťažkosti vnímať veci z pohľadu inej osoby.
Ku koncu štádia sú schopné lepšiemu uvedomeniu si rozdielu medzi fantáziou a skutočnosťou.
    \item Štádium konkrétnych operácií \\
Piaget tvrdil, že dieťa je v tomto štádiu približne od siedmeho do jedenásteho veku svojho života.
Dieťa si osvojuje stálosť a objavuje sa decentralizácia, teda schopnosť sústrediť pozornosť na niekoľko vlastností jedného objektu.
Začína rozumieť ekvivalencii\footnote{Ak sa v určitých vlastnostiach $A=B$ a $B=C$, tak sa aj $A=C$}, tranzitivite\footnote{Ak $A<B$ a $B<C$, tak $A<C$}, inverzii\footnote{$+A$ a $-A$ sú inverzné} či reciprocite\footnote{$A<B$ je reciprocné $B>A$}.
Dieťa uvažovať o časti a celku zároveň (schopnosť chápať inklúziu).
Ich myslenie je obmedzené na chápanie konkrétnu poznávaciu schopnosť, nie su schopné uvažovať o abstrakcii či o hypotetických podmienkach.
    \item Štádium formálnych operácii
Toto štádium sa začina dozretím v štádiu kognitívnych operácií.
Mal by to byť približne jedenásty alebo dvanásty rok života, ale rôzny odporcovia Piagetovej teórie tvrdia, že sa časť ľudí do tohto štádia ani nedostane.
Človek dosahujúci toto štádium je schopný uvažovať o potenciálnych a hypotetických situáciách, systematicky hľadať riešenia problémov, pravdepodobnostne uvažovať a abstraktne myslieť.
Rozvíja sa možnosť rozmýšľať o vlastných myšlienkach, hodnotách či morálnych princípoch.
\end{itemize}

Práve v poslednom štádium je úvodná časť života ľudí, pre ktorých je určený výsledok tejto práce konkrétne pre žiakov druhého stupňa. 
Napriek odporcom, o ktorých sme hovorili pri štádiu formálnych operácií, je potrebné v tomto veku utuženie si konkrétnych operácií a podchytiť rozvoj abstraktných operácií.
V našej práci týmito konceptami nie sú len štruktúry podmienka\footnote{príkaz $ak$} alebo zjednodušený for-cyklus\footnote{príkaz $opakuj$} či while-cyklus\footnote{príkaz $kym$}, ale aj procedúry s možnosťou rekurzie.
Sú existujú ľudia, ktorí nesúhlasia s učením rekurzie už v takto nízkom veku, ale autor si myslí, že prostredie Karel ponúka túto metódu v dostupnej forme a pokiaľ je správne odučená, žiaci sú jej schopný pochopiť už v tomto veku.


\section{Bloomova taxonómia}  \label{bloom}
%patdesiate roky mam odtialto: https://www.ff.umb.sk/mkmet/bloomova-taxonomia.html

V 50-tych rokoch publikoval Benjamin Samuel Bloom svoju taxonómiu, v ktorej rozdeľuje dimenzie kognitívneho procesu do šiestich stupňov. 

\begin{figure}[ht]
\centering
\missingfigure{Sem pôjde vektorovo nakresleny trojuholnik podla blooma - nechcem riesit (c), tak to nakreslim sam}
\caption{Bloomova taxonómia}
\end{figure}

Základná verzia prostredia Karel bola z pohľadu spomenutej taxonómie zameraná hlavne na druhý a tretí stupeň, teda pochopiť existujúci kód, ktorý predniesol vyučujúci a aplikovať poznatky pre zhotovenie ďalších riešení problémov.

Rozšírenie o možnosť ladenia v prostredí Karel podporuje lepšie trénovanie aj na štvrtom a piatom stupni teda analyzovanie a hodnotenie.
Žiak vďaka týmto nástrojom má uľahčený prístup k nahliadnutiu, ako sa správa existujúci kód a má zjednodušenú cestu k nájdeniu problému v kóde, ktorého nie je autorom.

\section{Podobné programy}  \label{podobne}

\todo[inline]{sem sa bude hodit taky prehlad nekarlovskych SW pre II. stupen. Imagine, scrach a tak. Vyhody+nevyhody + porovnat s Imaginom - skoda, ze sa neda rovnako ovladat ako robotnacka}

\section{Sú\texorpdfstring{č}{c}asné použitie a dôvody inovovania} \label{karelnagjh}

\todo[inline]{Povedat, ze su skoly, ktore karla pouzivaju od jeho vytvorenia v 97. Mala rozsirenost mohla byt dosledkom slabou propagaciou. Jedina bola pravdepodobne len na hodinach pre ucitelov na FMFI. Ze sme s Maru videli potencial, tak sme spravili vyskum a rozhodol som sa to preklopit a rozsirit o veci, ktore som z vyskumu povazoval za vhodne. Spomenut, ze sa pripravuje aj oficialne repo, aby to bolo lahsie na distribuciu a budeme aspon dohladatelny googlom. a samozrejme, ze sme opensource, visime aj na githube a ze sme otvoreni novym pull requestom s inovaciami. a nejaka mala propagacia za co najmensie zaklady - mozno odprezentovat na SVOC + FB}