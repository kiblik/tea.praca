\chapter{prostredie}
\todo[inline]{fixnut nazov kapitoly}

\section{História}

Názov prostredia vznikol podľa krstného mena divadelnej hry R.U.R.\footnote{česky Rossumovi Univerzální Roboti} Karla Čapka, ktorý bol tiež prvým človekom v histórii používajúci slovo Robot a to práve v spomenutej hre.
Programovací jazyk bol údajne\cite{karelnawiki} prvý krát vytvorený Richardom E. Pattisom v roku 1981 na Standfordskej univerzite v Kalifornii.
Vďačne používaným sa stal hlavne vďaka svojej jednoduchosti a prívetivej téme robotiky.

V rokoch 1987 a 1988 pre jeho sloveneskú mutáciu vychádzala dokonca séria článkov v časopise Zenit pionierov\cite{ZENIT}.
Bol určený pre vtedajší počítač PMD 85\footnote{Pieštanský Mikropočítač Displejový}.
Články boli písané hravou formou, aby boli blízke mladému čiteľovi. Vydavatelia dokonca zorganizovali v programovaní v tomto jazyku súťaž s možnosťou postúpiť do krajského a celoštátneho kola.

Karel od jeho vzniku prešiel procesom rôznych zlepšení.
Bol vytvorený tiež jazyk Karel++\footnote{\href{http://csis.pace.edu/~bergin/karel.html}{http://csis.pace.edu/\~{}bergin/karel.html}} alebo vznikla jeho implementácia formou kniznice pre rôzne vyššie programovacie jazyky ako Java\footnote{\href{http://web.stanford.edu/class/cs106a/book/karel-the-robot-learns-java.pdf}{http://web.stanford.edu/class/cs106a/book/karel-the-robot-learns-java.pdf}} alebo Python\footnote{\href{http://gvr.sourceforge.net/}{http://gvr.sourceforge.net/}}

\section{Porovnanie s ostatnymi}

\todo[inline]{
- nasiel som rozne verzie karla. malo z nich dava aj ulohy. tie, ktore ano, tak nie vzdy systematicky. z tych, odkial by som mohol aspon nieco cerpat:\\
-- MTSU: https://www.cs.mtsu.edu/~untch/karel/links.html\\
-- Stanford: http://web.stanford.edu/class/cs106a/book/karel-the-robot-learns-java.pdf\\
-- slabe, ale nieco male inspirativne: http://karel.oldium.net/priklady.html\\
-- IEEE: http://ieeexplore.ieee.org/stamp/stamp.jsp?tp=&arnumber=6554131\\
-- Masarykova uni: http://tutor.fi.muni.cz/index.php?p=training_solve&problem_id=6\\
-- Jaro/Mirka: https://vybo.gjh.sk/?page=karel\\
-- http://mormegil.wz.cz/prog/index.html \\
-- http://www.emil.input.sk/info_en.htm \\
-- http://gvr.sourceforge.net/links.php\\
-- https://csis.pace.edu/~bergin/karel/ecoop2000JBKarel.html\\
-- To co mala Basa na MUNI
}



\section{Preklopenie}

Nakoľko verzia Karla od Mgr. Ľuboša Košúta\cite{KOSUT97}bola dobre na naprogramovaná\footnote{skoro žiadna chybovosť, slovenská mutácia, prehľadný kód} a autor súhlasil nadviazaním na jeho projekte, rozhodlo sa, že sa bude stavať na uvedenej verzii. 
Spomenutý projekt bol od svojho publikovania v roku 1997 upravovaný Mgr. Jaroslavom Výbošťokom, pre potreby vyučovania na Gymnáziu Jura Hronca a Základnej škole Košícká\footnote{Aktuálne už ako jeden subjekt Spojená škola Novohradská}.

Vznikla požiadavka na vytvorenie nového kompilátu pre systém Microsoft Windows 64-bitovej verzie, nakoľko tam pôvodnú verziu nebolo možné spustiť.
Okrem toho sa požadovalo skompilovanie pre operačný systém Linux.
Na základe požiadaviek sa uznieslo, že bude potrebné ,,preklopenie'' projektu z prostredia Delphi do prostredia Lazarus založenom na kompilátore freepascal.

Prostredie takéto preklopenie natívne podporuje avšak s určitými obmedzeniami.
Bolo potrebné dodatočné opravenie gramatiky, nakoľko išlo o slovenské program.


\todo[inline]{zvazit ci este nieco neplnit, ale ved koho zaujmaju programatorske zvasty o tom, co som musel fixovat. hlavne, ze to funguje}

\section{Funkcionalita}
\todo[inline]{
popis prostredia\\
narabanie s projektom \\
zmeny prostredia \\
prikazy \\
spomenut, ze povodna verzia mala highlighter a preco sa odstranil \\
pozriet este Lubovu pracu, ci sa neoplati nieco konkretne z nej citovat \\
zvazit a zdovodnit pridanie prikazu vpravo \\

}
\section{Pridaná funkcionalita}

Hlavným prínosom práce, okrem metodiky pre učiteľov, pridaná funkcionalita a oprava niekoľkých chýb, o ktorých nás vyučujúci počas výskumu informovali alebo tie, na ktoré si prišli sami. 

\subsection{Zadania}

Z výskumu vzišla požiadavka o možnosť tvorby zadaní.
Funkcionalita by mala vyučujúcemu zjednodušiť prácu pri príprave a zadávaní úloh, ktoré majú žiaci plniť.
Prínos je cítiteľný aj v prípade, že je študent neprítovný.
Vyučujúci má zjednodušenú možnosť, ako žiakovy ukázať aj požiadavku na finálny výsledok.

Funkciu zadaní nie je potrebné samostatne zapínať, nakoľko je automaticky dostupná.

Jeden projekt môže obsahovať viac zadaní.
Každé zadanie má svôj názov, slovný popis a ukážku riešenia.

V pravá časť prostredia je rozdelená na 3 časti.

\begin{itemize}
    \item Slovený popis zadania \\
Žiak si môže prečítať, popis problému, ktorý mu zadal učiteľ.
    \item Zobrazenie riešenia \\
Kliknutím na tlačidlo sa študentovi zobrazí samostatné okno, v ktorom je ukážka rozmiestnenia tehál, značiek a karla v miestnosti tak, ako to učiteľ od žiaka požaduje.
    \item Zoznam zadaní/problémov \\
V liste si môže žiak zvoliť zadanie, ktoré chce riešiť.
Ak žiak prepíma medzi zadaniami, riešenie každého zostáva uložené v stave, v akom ho zanechal a nevracia sa to inicializačnej polohy.
\end{itemize}
\todo[inline]{domysliet ci nechcem pridat aj reset zadania, aby ked to poondi mohol lahko zacat od zaciatku - asi to spravime, len treba domysliet implementaciu}

\begin{figure}[ht]
\centering
\missingfigure{Sem pôjde screenshot z prostredia, kde je niekolko zadani}
\caption{Zadania}
\end{figure}

Príkazy, ktoré žiak vytvoril v jednom zadaní sa zdieľajú a môže ich opäť využiť pri riešení ďalšieho zadania.
Odporúčame učiteľom, aby v zadaniach žiakom povedali, aby rozumne pomenovali ich riešenia, napríklad: ,,prikaz riesenie1''.

Prostredie na tvorbu zadaní je implementované priamo v hlavnom prostredí.
Učiteľ si možnosť editácia odomkne stlačením tlačidla Pause\todo{toto mozno este zmenime}.

Od tohto momentu môže zadávateľ robiť úpravy zadaní.
Slovné popisy sa upravujú priamo v mieste, kde sa žiakom zobrazoval slovný popis.
Vkladanie zadaní, ich premenovanie a mazanie je možné prostredníctvom pravého kliku v zozname zadaní.
Úvodná konstelácia a ukážka riešenia sa realizuje priamo v prostredí, kde žiaci bežne riešia problémy.
Sú na to použité aj rovnaké príkazy.
Ak má učiteľ záujem namiesto editácie východzej konštelácie upravovať ukážku riešenia, využije tlačidlo\todo{teraz neviem, ako sa vola}, ktoré je situované na mieste, z ktorého si žiaci zobrazovali zadanie.
Pri jeho stlačení sa úvodná konštelácia skryje na jej mieste sa zobrazí požiadavka na riešenie problému.
Opätovným stlačením sa učiteľ znova dostane späť k editácii úvodného rozloženia.

\todo[inline]{zvazujem pridat pre ucitela aj priamy rezim, lebo sa obavam, ze sa uprikazuje, kym vytvori zadanie. na druhu stranu by si aspon overil, ze jeho zadanie ma skutocne riesenie ;)}

\subsection{Prostredie na ladenie}

Ďalším rozšírením je možnosť ladenia\footnote{anglicky debugging} respektíve krokovania.
Do tohto režimu je možné sa dostať dvoma spôsobmi:
\begin{itemize}
    \item Pozastavením počas vykonávania \\
Prostredie doteraz umožňovalo zastaviť proces vykonávania stlačením tlačidla stop.
V tomto prípade sa všetka vykonávaná akcia pozastavila a nebolo mžné v nej porkačovať.
K tomuto tlačidlu bolo pridané druhé pozastav. \todo[]{este zvazujem zmenu nazvu a spravenia peknej ikonky}
    \item Pozastavenie prostredníctvom jazyka \\
Do jazyka prostredia bol pridaný príkaz pozastav. \todo[]{este zvazit zmenu, ale asi nie. toto je vyztizne}
Zavolanie tohto príkazu je ekvivalentné stlačeniu tlačidla pozastav.
V bežne používaných programovacích jazykoch by sme nazvali túto ,,značku'' ako breakpoint.
Nevýhohou je, že k pozastaveniu dôjde vždy narozdiel od bežných programovacích prostredí, kde je možné breakpoint počas behu programu odstraniť.
\end{itemize}

\begin{figure}[ht]
\centering
\missingfigure{Sem pôjde screenshot tlacidiel resp. skor ladiaceho okna}
\caption{Tlačidlá}
\end{figure}

Vykonaním tejto pozastavovacej akcie sa anuluje príkaz rýchlo, lebo by nebolo možné skúmať vykonávaný krok. \todo[]{tu este zavujem, ci po zruseni pozastavenia sa vratit aj prikaz rychlo do povedneho stavu - asi by sa hodilo}

Od momentu pozastavenia sa žiak nachádza v móde ladenia.
Môže v ňom pokračovať po krokoch\footnote{V tomto kontexte je vhodnejšie použiť spojenie ,,po príkazoch'', nakoľko slovo krok je rezervovaným slovom jazyka Karel pre vykonanie posunu vpred} ďalej.

Návrat do režimu štandardného pokračovanie je možný dvoma spôsobmi:
\begin{itemize}
    \item Stlačením tlačidla pokračuje \\
Toto tlačidlo je situované na ladiacom panely\footnote{zatial je na podovnom mieste, treba ho presunut}
    \item Použitím príkazu pokračuj \\

\end{itemize}
\todo[inline]{Stojí za zváženie odstránenie tohto príkazu. Ved ani bezne nemas ani breakponty v programoch}
\todo[inline]{zvazujem zobrazovanie aktualnej casti kodu - treba domysliet, ako a tiez v pripade opakovani cislo iteracie. plus by mohol zobrazovat, ako by sa pre aktualnu situaciu vyhodnotili podmienky: stena, volno, tehla, znacka}
\todo[inline]{ani aktualne vchadzanie do ladenia mi neprijde az tak fajn. po pauznuti to dam asi do samostatneho okna}
\todo[inline]{zmenit formulaciu zaverecnej hlasky. co sa zlepsilo od 90tych rokov? vsetko je zalozene na intuicii. ano treba ziakov ucit k doslednosti citania, ale myslim si, ze by sa dal najst aj lepsi sposob, ako narusovat userexperience o Karlovi ;) }
\todo[inline]{pridane nie aj resetnutie zadania, aby sa dalo skusat = ziak si to pokazi, treba ist odznovu}