\chapter{Metodika u\texorpdfstring{č}{c}enia}

Programovací jazyk v prostredí Karel je jednoduchý, ale má silu Turingovho stroja\todo{bolo by dobre preverit a nejak jednoducho dokazat}.
Pri konštrukcií nie je možné využiť premenné, ale vďaka rekurzii je možné určité zapamätanie si počtov.
Učiteľ by pri tvorbe úloh pre žiakov nemal zabúdať na tieto vlastnosti a mal by príslušne k tomu upraviť náročnosť úloh.

Neskôr si ukážeme, že existujú typy úloh, ktoré bude možné riešiť cyklami, ale použitím rekurzie sa nám riešenie podstatne zjednoduší.
V nasledujúcej kapitole prejdeme jednotlivými konštrukciami, ktoré prostredie ponúka a ukážeme si vhodné úlohy, ktoré je možné použiť pre výučbe.
Budeme postupovať od jednoduchších k zložitejším tak, ako by ich bolo vhodné použiť pri učení. 

Vhodné úlohy pre prostredie Karel boli publikované v Časopise Zenit\cite{ZENIT}.
Edukácia bola mladým čitateľom priblížené prívetivej podobe a časopis dokonca organizoval programátorskú súťaž v jazyku Karel.
Určitá časť úloh bude uvedená v jednotlivých častiach metodickej príručky.

V ďalšej časti budeme používať pojem vlastný príkaz\footnote{resp. príkaz}.
V bežnej praxi sa tento pojem zamieňa tiež sa pojmy procedúra alebo funkcia\footnote{v závislosti od kontextu}, ale pre zjednodušenie pochopenia princípu žiakom druhého stupňa sme uprednostnili tento pojem.

\section{Uvodne}

\todo[inline]{rozumne povenovat sekciu, pripadne pridat dalsiu a naplnit obsahom}

\section{Rekurzia}

Rekurzia je pre žiakov základnej školy na pochopenie netriviálny koncept.
Nie je však nezvládnuteľný.
Niektorí učitelia sa rozhodnú ho neučiť, čo je podľa autora nevhodné a pokiaľ im to dovoľuje časová dotácia ich vyučovacích hodín, mali by túto látku do procesu zaradiť.

Štátny vzdelávací program\cite{ISCED2} neprikazuje učenie tohto konceptu a zostáva pri premenných, cykloch či procedúrach.
ŠVP je dokument určujúci minimálny obsah učiva avšak rozšírenie o rekurziu je prakticky žiadúce\footnote{podľa názoru autora práce}.

Rekurziu je možné žiakom popísať ako príkaz, ktorý volá samého seba. Praktickým príkladom môžu byť žiakom napríklad matriošky\footnote{duté drevenené figúrky skoro valcovitého tvaru rôznej veľkosti, ktoré je možné vzájomne do seba vložiť}.
Pre programatórske účeli sa hodí vizuálna podoba prostredníctvom obrázku \ref{priamarekurzia}, kde je znázornená priamá rekurzia.

\begin{figure}[ht]
\centering
\missingfigure{Sem pôjde obrázok priamej rekurzie}
\caption{Priama rekurzia}
\label{priamarekurzia}
\end{figure}

Poznáme samozrejme aj nepriamu rekurziu.
V tomto prípade nejde o volanie ,,samého seba'', ale o nepriame volanie prostredníctvom ďalšieho príkazu (alebo viacerých príkazov) - viď obrázok \ref{nepriamarekurzia}.

\begin{figure}[ht]
\centering
\missingfigure{Sem pôjde obrázok nepriamej rekurzie}
\caption{Nepriama rekurzia}
\label{nepriamarekurzia}
\end{figure}

Rekurziu môžeme najsť aj v iných programovacích jazykoch učených pre deti.
Príkladom je jazyk Logo-vského typu v prostredí Imagine alebo Scratch.

Poznáme rekruzie troch typov a postupne prejdeme cez jednotlivé typy.

\subsection{Nekone\texorpdfstring{č}{c}ná rekurzia}

Už z názvu je zrejmé, že ide o rekurziu, ktorej koniec nie je podmienený jazykom Karela. Je ale limitované prostredím Karel\footnote{rekurzia sa implementuje prostredníctvom zásobníka a ten nie je nekonečný}. Žiakom prívetivým príkladom môže byť kvetinka, ktorá sa objavuje a mizne z príkladu \ref{kvetinka}

\begin{lstlisting}[label=kvetinka,caption=Kvetinka,captionpos=b]
prikaz a
  ak je tehla rob
    zober
  inak
    poloz
  *ak
  a
*prikaz
\end{lstlisting}

Tiež je žiakom možné túto rekurziu ilustrovať na príklade spomenutom aj v \cite{Snajder03} a to pesničkou: ,,Byl jeden Číňánek, měl hezký copánek, vyskočil na tramvaj \ldots a na hrob napsala ,,Byl jeden Číňanek '' \ldots ''.
Ide o pesničku založenú na neustálom opakovaní rovnakého celku, pričom záverečný verš odkazuje na ďalšie spievanie toho istého celku, nakoľko tematicky nadväzuje.
Celý text je uvedený aj v prílohe \ref{priloha:cinanek}.

Vrámci trénovania štrvtého stupňa Bloomovej taxonímie\cite{bloomtax} by bolo vhodné zanalyzovať so žiakmi príklady \ref{nekonecna_zaciatok} a \ref{nekonecna_koniec}.

\begin{lstlisting}[label=nekonecna_zaciatok,caption=Nekonečná rekurzia s volaním na začiatku,captionpos=b]
prikaz b
  b
  vlavo
*prikaz
\end{lstlisting}

\begin{lstlisting}[label=nekonecna_koniec,caption=Nekonečná rekurzia s volaním na konci,captionpos=b]
prikaz c
  vlavo
  c
*prikaz
\end{lstlisting}

V posudzovaní príkladu je potrebné, aby si žiaci uvedomili, že kódy programov sú veľmi podobné, ale budú mať odlišné správanie.
Zatiaľ čo v prvom prípade sa Karel nepohne, pretože bude vždy vnáraný sám do seba, v druhom bude nastávať otáčanie, lebo vnáranie nastáva až po zavolaní príkazu vlavo.

Z praktického hľadiska je potrebné učiteľov upozorniť, že aplikácia je naprogramovaná formou zásobníka a pri každom volaní vlastného príkazu si odloží referenciu na miesto, kam s má vrátiť.
Tento druh implementácie môže spôsobiť preplnenie zásobníka a následne pád programu.
Keďže veľkosť operačnej pamäte je veľká, napĺňanie zásobníka bude trvať netriviálny čas, ale pri zanechaní programu bežať ho dostatočne dlho, dôjde k úplne preplneniu.

\subsection{Vchostová rekurzia}

O chvostovú rekurziu ide v prípade, že volanie sa nachádza na konci procedúry. Ide o typ konštrukcie, ktorá je zapísateľná aj v nerekurzívnej forme. Preto sú nasledujúce dva príklady z \cite{Snajder03} ekvivalentné.

\begin{lstlisting}[label=chvostova_rekurzia,caption=Chvostová prekurzia,captionpos=b]
prikaz y
  ak nie je stena rob
    krok
    y
  *ak
*prikaz
\end{lstlisting}

\begin{lstlisting}[label=eq_chvostova_rek,caption=Nerekuzívne riešenie problému \ref{chvostova_rekurzia},captionpos=b]
prikaz x
  kym nie je stena rob
    krok
  *kym
*prikaz
\end{lstlisting}


\todo[inline]{napad na zlepsenie prostredia = mato vobec vyznam?

Podla Blaha – pripisanie celeho tela na koniec
}

\subsection{Pravá rekurzia}
\todo[inline]{ved vieme co tu treba}

\section{narocne ulohy}

\todo[inline]{
- po skonceni zakladnych ulohy, hodili by sa aj zaujmave. Prve napady: najdi cestu v bludisku - pre mladsich pravidlo pravej/lavej ruky, pre starsich prehladavanie do hlbky, rozlozenie 8 dam/znaciek na sachovnici}

\todo[inline]{ulohy zo zenitu:
ulohy:
- cvz
- postav stlp
- urob chodnik
- chod po moste - nespadni do vody
- invertuj rad
- objavujuca sa a miznuca kvetinka
- prejdi cez mur - konkretna vyska, lubovolna vyska(otvorit si miestnist, preniest stlp, zatvorit miestnost)
- pohyb po papiery
- studna
- studna bez stupania na policka
- dlazdicvoanie miestnosti - po radoch, eSicko alebo spirala, 3 rady(vlavo, podseba, vpravo)
- sachovnica
- 
}